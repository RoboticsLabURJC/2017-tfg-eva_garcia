\chapter{Prácticas con Scratch}
\label{cap:scratch}
En este capítulo, se explicará cómo y por qué se diseñaron los distintos ejercicios con el lenguaje de programación educativo Scratch, haciendo hincapibooké en la finalidad de cada uno de ellos. \\
El capítulo se estructurá en varias secciones, una por cada área, u objetivo, de enseñanza: actuadores, sensores y comportamientos. Para cada sección, se explicarán los ejercicios que se han diseñado, con los propósitos que se buscaban con ese diseño (tanto para la robótica como para la programación en general) y la validación experimental que se ha hecho con los ejercicios en un entorno real.\\
El primer apartado hablará de los actuadores. La definición de un actuador es un componente que recoge datos de la placa (del programador, por tanto) y \textit{actúa} de acuerdo a esos datos. En ningún momento devuelve ningún dato a la placa. Por esta razón, la programación de este tipo de componente es más sencilla por ser más intuitiva: el alumno manda hacer algo al robot y el robot lo hace, sin necesidad del tratamiento de datos del entorno. Precisamente, al ser la programación del componente más sencilla en concepto, el diseño de ejercicios acepta más complejidad en conceptos de programación (conceptos como bucles, condiciones, esperas, etc). debido a esto se eligieron los actuadores como primer área para diseñar las prácticas.\\
La segunda sección de este capítulo se ocupará de los sensores. Un sensor recoge datos del entorno y los devuelve, en formato numérico, a la placa. Esto requiere de más control y estructuración de las ideas que los actuadores, pues requiere del programador llamar al sensor, guardar la información, tratarla de forma adecuada y, por último, programar la actuación del robot de acuerdo a ella. Por tanto, y contando con la base de los conocimientos y entrenamiento que habrían proporcionado los ejercicios del capítulo anterior, los sensores se escogieron como segundo área de conocimiento.\\
La tercera y última parte se centrará en la integración de los dos anteriores, es decir, en la integración de sensores y actuadores para la solución de problemas reales (todo lo reales que es posible, dadas las limitaciones del robot). Este tipo de ejercicios los llamamos comportamientos, y se han diseñado de tal forma que acepten escalonamiento en la dificultad. Esto responde a la necesidad de combatir la frustración de los alumnos más jóvenes, pues si se les presenta un problema sin dividir en problemas más pequeños, no sabrán cómo atajarlo. Obviamente, esta división del problema no se les dará a los alumnos directamente, sino que se les guiará para que la hagan ellos con el propósito de introducirlos en la programación \textit{bottom-up}.
























