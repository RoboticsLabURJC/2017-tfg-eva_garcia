\chapter{Objetivos}
\label{cap:objetivos}

Como se ha adelantado en la introducción, el carácter, y por tanto el objetivo, principal de este Trabajo Fin de Grado es educacional. Siguiendo este carácter, abordaremos dos caminos complementarios, explicados a continuación. 
\\
	\par \textbf{Programación en Python de un robot basado en Arduino}. El objetivo técnico será proporcionar una biblioteca en python que poder utilizar para programar un robot concreto basado en una placa base de Arduino, con el fin de dar una opción de lenguaje más sencilla. Para que esto sea posible, también se trabajará en un programa residente, en Arduino, que grabar en la placa base, que ofrezca una comunicación con la biblioteca de python. Así, se podrán programar los sensores y actuadores del robot con funciones en python, cuya lógica estará en esta biblioteca y de la cual no tendrán que preocuparse los alumnos. 
	\\
	\par \textbf{Propuesta educativa completa, basada en Scratch y Python, y escalonada según dificultad}.
	Ofreceremos una propuesta educativa, para un curso escolar, orientada según niveles de dificultad y con objetivos docentes. Para ellos, crearemos diferentes ejercicios, o prácticas, de robótica -con el robot educacional Mbot, describiendo los objetivos conceptuales que se persiguen y ordenándolas con la finalidad de un aprendizaje gradual de programación. Este curso estará orientado principalmente a alumnos de Educación Primaria o Secundaria (alumnos sin conocimientos previos de programación). \\
	Para esta propuesta, utilizaremos tanto el lenguaje de programación por bloques Scratch, proporcionado por el fabricante, como nuestro \textit{midleware} en Python, más complejo y, por tanto, como segunda parte avanzada.
