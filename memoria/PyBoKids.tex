\chapter{Plataforma PyBo-Kids}
\label{cap:PyBoKids}
En este capítulo explicaremos el proceso seguido para desarrollar las bibliotecas Arduino y Python, explicando el código y las distintas necesidades que han surgido durante el desarrollo. \\
Como se explicó en el Capítulo \ref{cap:infra}, se ha utilizado Arduino  y su IDE nativo para esta programación, y Python 3 y un editor de texto estándar (en este caso, Visual Studio Code, de Microsoft) para la programación en Python. 


\section{Programa residente}\label{sec:residente}
El primer paso necesario en 
\section{Programa PC}
\section{Diseño}\label{sec:diseño}
\section{Resultado: programa PC y programa residente}\label{resultado}