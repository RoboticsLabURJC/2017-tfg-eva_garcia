\chapter{Introducción}
\label{cap:introduccion}

\section{Contexto y motivación}
Durante las últimas décadas, y en especial durante los últimos años, la Tecnología ha vivido un crecimiento inmenso en todas sus áreas; televisión, telefonía, comunicaciones, incluso en las artes. El área que ocupa a este Trabajo Fin de Grado, la robótica, ha sido quizá uno de los más espectaculares, ya que era también uno de los menos estudiados y desarollados. Hace cuarenta años la máxima tecnología eran los automatismos en la industria, y ahora tenemos robots que ayudan a la comunicación de personas con TEA.\\
Esta rama ha crecido tanto que se ha convertido casi en un área de conocimiento propia, por lo que existe una gran demanda de profesionales dedicados a ello. Las universidades se han hecho eco de esa necesidad, y han incluido másteres, especialidades, e incluso asginaturas para orientar a los alumnos que quieran dedicarse a la Robótica e investigación. Sin embargo, ese eco no ha llegado a etapas más bajas de la educación, y nos encontramos a alumnos que, queriendo dedicar su carrera a la investigación en Robótica, no tienen conocimientos previos, ni teóricos ni prácticos, en los que apoyarse. Existe, por tanto, una gran necesidad educativa en iniciación a la Robótica. Este Trabajo Fin de Grado, y el proyecto JdeRobot en general, tratarán de dar salida a esa necesidad. \\
El mayor desafío a la hora de enseñar Robótica es que los entornos, los lenguajes, las herramientas en general no son muy amigables para alguien que nunca antes haya desarrollado software. En este caso, además, sumamos el hándicap de la edad de los alumnos; es muy díficil mantener la concentración de alumnos de 11-15 años durante dos horas delante de un ordenador. El objetivo era, por tanto, armar un entorno amigable, que aceptara 