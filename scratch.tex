\chapter{Prácticas con Scratch}
\label{cap:scratch}
En este capítulo, se explicará cómo y por qué se diseñaron los distintos ejercicios con el lenguaje de programación educativo Scratch, haciendo hincapibooké en la finalidad de cada uno de ellos. \\
El capítulo se estructurá en varias secciones, una por cada área, u objetivo, de enseñanza: actuadores, sensores y comportamientos. Para cada sección, se explicarán los ejercicios que se han diseñado, con los propósitos que se buscaban con ese diseño (tanto para la robótica como para la programación en general) y la validación experimental que se ha hecho con los ejercicios en un entorno real.\\
El primer apartado hablará de los actuadores. La definición de un actuador es un componente que recoge datos de la placa (del programador, por tanto) y \textit{actúa} de acuerdo a esos datos. En ningún momento devuelve ningún dato a la placa. Por esta razón, la programación de este tipo de componente es más sencilla por ser más intuitiva: el alumno manda hacer algo al robot y el robot lo hace, sin necesidad del tratamiento de datos del entorno. Precisamente, al ser la programación del componente más sencilla en concepto, el diseño de ejercicios acepta más complejidad en conceptos de programación (conceptos como bucles, condiciones, esperas, etc). debido a esto se eligieron los actuadores como primer área para diseñar las prácticas.\\
La segunda sección de este capítulo se ocupará de los sensores. Un sensor recoge datos del entorno y los devuelve, en formato numérico, a la placa. Esto requiere de más control y estructuración de las ideas que los actuadores, pues requiere del programador llamar al sensor, guardar la información, tratarla de forma adecuada y, por último, programar la actuación del robot de acuerdo a ella. Por tanto, y contando con la base de los conocimientos y entrenamiento que habrían proporcionado los ejercicios del capítulo anterior, los sensores se escogieron como segundo área de conocimiento.\\
La tercera y última parte se centrará en la integración de los dos anteriores, es decir, en la integración de sensores y actuadores para la solución de problemas reales (todo lo reales que es posible, dadas las limitaciones del robot). Este tipo de ejercicios los llamamos comportamientos, y se han diseñado de tal forma que acepten escalonamiento en la dificultad. Esto responde a la necesidad de combatir la frustración de los alumnos más jóvenes, pues si se les presenta un problema sin dividir en problemas más pequeños, no sabrán cómo atajarlo. Obviamente, esta división del problema no se les dará a los alumnos directamente, sino que se les guiará para que la hagan ellos con el propósito de introducirlos en la programación \textit{bottom-up}.

\section{Actuadores}
Como se ha explicado anteriormente, el propósito de un actuador es ejecutar la orden que recibe de la placa. Es, por tanto, muy visual y conveniente para introducir en la programación de robótica, pues da pie a equivocarse y aprender de los errores de forma mucho más intuitiva que con otros componentes.\\
En la sección \ref{sec:mbot}, ya se detallaron los actuadores con los que cuenta el Mbot. En el diseño de estas prácticas no se han utilizado todos, pues se contó con la limitación de tener sólo el paquete "básico" del mbot (presumiblemente, sería el que tendrían los centros educativos). Estos componentes sería, pues, los motores, las luces led integradas en la placa, la placa led, el zumbador integrado y el pulsador integrado.\\
A continuación se explicarán los ejercicios desarrollados, intentando explicar el objetivo del ejercicio, el nivel de dificultad y comentando la experiencia de llevarlo a cabo.
\subsection{Carreras}
Obviamente, lo más visual del robot Mbot son los motores (las ruedas). También es lo más intuitivo y esperable de un robot en general: se mueve por sí mismo. Es uno de los componentes más sencillos de programar, ya que Scratch ofrece un desplegable con los posibles valores que aceptan los motores del Mbot. Además, tiene métodos directamente para avanzar, retroceder y girar, muy cómodos para el primer contacto con el robot y válidos para este ejercicio en concreto. \\
En vez de simplemente mover el robot, pensamos el ejercicio como una carrera entre los robot de todos los alumnos para despertar su interés y hacerlo más entretenido. Esto se comprobó necesario en la práctica; sin una utilidad práctica, o juego, los alumnos perdían interés rápidamente o, incluso, no lo tenían desde el principio. \\
\subsubsection{Ejercicio extra}
Otra cosa que se pudo comprobar cuando se pusieron en práctica los ejercicios es que es mejor hacer varias versiones de un mismo ejercicio, aunque la dificultad sea la misma, con objetivos distintos. Los alumnos no pierden la atención y aprenden que una misma solución, con pequeñas variaciones, vale para distintas situaciones. Además, así tenemos cubierta la situación de que los alumnos vayan a distintas velocidades y unos terminen el ejercicio base mientras otros no.\\
Una versión del ejercicio de las carreras es hacerla con obstáculos fijos. Los alumnos tendrán que medir en tiempo la distancia entre obstáculos y jugar, con las velocidades y los giros. Quién mejor combinación entre ello encuentre, mejor tiempo hará en carrera.

\subsection{Cumpleaños feliz}
Otro actuador sencillo de usar es el zumbador integrado en la placa. Es algo más complicado, pues las notas musicales no son las más conocidas y, sobre todo en los alumnos más jóvenes, no están acostumbrados a una escala musical. Sin embargo, no puede comprobar que no tardan mucho en entender el sistema de escalas; se les puede ayudar dándoles una tabla de equivalencias entre las notas americanas y europeas. 
\begin{figure}[H]
	\centering
	\includegraphics[width=0.7\textwidth]{img/lorem.jpg}
	\caption{Equivalencia entre notas}	
\end{figure}
Una vez entendida la diferencia de notas y comprobado cómo funciona el zumbadores, les proponemos que el robot entone la canción de cumpleaños feliz -por ser una canción que todo el mundo conoce-. No sólo sirve para el actuador, sino que introducimos el concepto de repetición (bucles). La idea principal es dejarles al principio que hagan el ejercicio todo seguido y después les ayudamos a entender que están repitiendo código: hay estrofas que se repiten. Ven la necesidad, pues, de meter esas estrofas dentro de un bucle de repetición de tantas veces como sea necesario.\\
\subsubsection{Ejercicio extra}
Como ocurrió antes con las carreras, una vez han terminado la primera canción, es bueno invitarles a que intenten que el robot "cante" otras melodías que conozcan y les apetezca. 

\subsection{S.O.S.}
El siguiente actuador del pack del Mbot son las luces LED de la placa, que tienen un rango de colores RGB que permite que cada alumno pueda personalizar los colores como más les guste. El propio lenguaje Scratch permite el rango de cada color entre 0 y 255 (aunque en la práctica, no sean notables los cambios pequeños en esa escala).\\
La finalidad del ejercicio es que, con luces, el robot pida ayuda en lenguaje morse: tres pulsos cortos, tres pulsos largos, y tres pulsos cortos. \\
Por experiencia, son más proactivos cuando ven una utilidad mínimamente real a lo que se les pide, y en muchas películas han visto hablar del morse y la necesidad de pedir ayuda en un lenguaje universal. \\
Como concepto de programación, en este ejercicio trabajamos las esperas. Los puntos y rayas del morse los traducimos a esperas de menos y más tiempo, respectivamente.
bucle de repetición de tantas veces como sea necesario.\\
\subsubsection{Ejercicio extra}
Como siempre, una vez que han conseguido que el mensaje de socorro sea entendible, cambiamos el ejercicio a escribir en morse cualquier mensaje que se les ocurra (con la tabla de equivalencias del abecedario) y a mandar el mismo mensaje de socorro con luz y sonido a la vez.

\subsection{Camión}
Como último ejercicio de actuadores, intentaremos juntarlos todos, a la vez que todavía no metemos ningún sensor. Proponemos, pues, emular el comportamiento de un camión, algo a lo que están acostumbrados y que no se les habría ocurrido que era automatizable. Un camión, haga lo que haga, si da marcha atrás, pitará (con los zumbadores) y encenderá las luces naranjas. \\
Vamos metiendo, poco a poco, acciones simultáneas, para que tengan que preocuparse de más de un componente (de más de un bloque de Scratch). El objetivo es el aprendizaje de que, en robótica, un comportamiento aparentemente trivial es la suma de muchos comportamientos simultáneos y dependiente uno de otro.