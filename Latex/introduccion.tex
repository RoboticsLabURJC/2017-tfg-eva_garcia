\chapter{Introducción}
\label{cap:introduccion}

\section{Contexto y motivación}
Durante las últimas décadas, y en especial durante los últimos años, la Tecnología ha vivido un crecimiento inmenso en todas sus áreas; televisión, telefonía, comunicaciones, incluso en las artes. El área que ocupa a este Trabajo Fin de Grado, la robótica, ha sido quizá uno de los más espectaculares, ya que era también uno de los menos estudiados y desarollados. Hace cuarenta años la máxima tecnología eran los automatismos en la industria, y ahora tenemos robots que ayudan a la comunicación de personas con TEA.\\
Esta rama ha crecido tanto que se ha convertido casi en un área de conocimiento propia, por lo que existe una gran demanda de profesionales dedicados a ello. Las universidades se han hecho eco de esa necesidad, y han incluido másteres, especialidades, e incluso asginaturas para orientar a los alumnos que quieran dedicarse a la Robótica e investigación. Sin embargo, ese eco no ha llegado a etapas más bajas de la educación, y nos encontramos a alumnos que, queriendo dedicar su carrera a la investigación en Robótica, no tienen conocimientos previos, ni teóricos ni prácticos, en los que apoyarse. Existe, por tanto, una gran necesidad educativa en iniciación a la Robótica. Este Trabajo Fin de Grado, y el proyecto JdeRobot en general, tratarán de dar salida a esa necesidad. \\
El mayor desafío a la hora de enseñar Robótica es que los entornos, los lenguajes, las herramientas en general no son muy amigables para alguien que nunca antes haya desarrollado software. En este caso, además, sumamos el hándicap de la edad de los alumnos; es muy díficil mantener la concentración de alumnos de 11-15 años durante dos horas delante de un ordenador. El objetivo era, por tanto, armar un entorno amigable, que aceptara distintos niveles de dificultad, y que utilizara un lenguaje con una sintaxis, si no entendible, al menos sí asumible para los estudiantes más jóvenes. 
\section{Proyecto JdeRobot}
El proyecto JdeRobot-Kids nació como una rama del proyecto de robótica educativa JdeRobot, orientado a los alumnos más jóvenes. Surgió en respuesta a esa necesidad de preparar y predisponer a los jóvenes. \\
Durante el desarrollo del proyecto han surgido varias cuestiones a las que se ha dado respuesta de la mejor forma posible. Por ejemplo, qué lenguaje \textit{amigable} utilizar, o cómo montar la infraestructura, o qué robot sería mejor para usar como base de los ejercicios. 
\subsection{Lenguaje}
Como casi todas las placas base de los robot utilitarios actuales utilizan Arduino, y siendo éste un lenguaje con una sintaxis muy cerrada y compleja como para utilizarla en un entorno de desconocimiento de programación, se pensó en programar en Python. Éste es un lenguaje con una sintaxis mucho más cómoda que otros, las palabras reservadas son entendibles -siempre en inglés, por supuesto- y a la vez es orientado a objetos, lo cual daba pie a enseñar conceptos de programación más complejos de una forma muy intuitiva. Sin embargo, ¿cómo programar en Python un robot cuya placa base está preparada para ser programada en Arduino? \\
La solución fue el middleware JdeRobot, una suerte de traductor entre la placa base (mensajes en hexadecimal y actuadores y sensores), y la infraestructura en Python, la que finalmente se utilizaría para programar los ejercicios reales del robot. 