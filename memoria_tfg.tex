\documentclass[a4paper, 12pt]{book}


\usepackage[a4paper, left=2.5cm, right=2.5cm, top=3cm, bottom=3cm]{geometry}
\usepackage{times}
\usepackage{afterpage}

\usepackage[utf8]{inputenc} %¡¡NO codificar en latin!!
\usepackage[spanish]{babel}
\usepackage{enumerate}
\usepackage{url}
%\usepackage[dvipdfm]{graphicx}
\usepackage{graphicx}
\usepackage{float}  %% H para posicionar figuras
\usepackage[nottoc, notlot, notlof, notindex]{tocbibind} %% Opciones de índice
\usepackage[none]{hyphenat}
\usepackage{cite}
\usepackage[pagestyles]{titlesec}
\usepackage{lipsum}
\usepackage[bookmarks = true, colorlinks=true, linkcolor = black, citecolor = black, menucolor = black, urlcolor = blue]{hyperref} %para que los enlaces no aparezcan recuadrados en rojo y cambia los colores de los enlaces

%para que el índice llegue hasta subsection
\setcounter{tocdepth}{2}
\setcounter{secnumdepth}{3}

\begin{document}

	\begin{titlepage}
		\centering
		\includegraphics[scale=0.5]{img/logo_urjc.jpg}
		\vspace{3cm}
		
		\Large
		GRADO EN INGENIERÍA EN TECNOLOGÍAS DE LA TELECOMUNICACIÓN
		
		\vspace{0.4cm}
		
		\large
		Curso Académico 2018/2019
		
		\vspace{0.8cm}
		
		Trabajo Fin de Grado
		
		\vspace{2.5cm}		
		\LARGE		
		PRÁCTICAS EDUCATIVAS DE ROBÓTICA CON MBOT EN ENTORNO JDEROBOT-KIDS
		
		\vspace{2.5 cm}
		
		\large
		Autor : Eva García Domingo\\
		Tutor : Jose María Cañas Plaza
		\afterpage{\null\newpage}
		\pagestyle{empty}
	\end{titlepage}
	
	
\thispagestyle{empty}
\begin{flushright}
	\textit{Dedicado a \\
		mi familia, amigos, compañeros de estudios.}
\end{flushright}
\afterpage{\null\newpage}
\pagestyle{empty}

	\input{agradecimientos}
	\chapter*{Resumen}
\thispagestyle{empty}
\label{cap:resumen}



\afterpage{\null\newpage}
	
	\tableofcontents
	\pagenumbering{arabic}
	\setcounter{page}{1}
	\mainmatter

	\chapter{Introducción}\label{cap:introduccion}

\section{Contexto}\label{sec:contextoIntroduccion}
Durante mucho tiempo la robótica ha sido una disciplina muy alejada y poco accesible al público, tanto por su coste económico, como por el difícil acceso a recursos académicos (manuales teóricos o recursos humanos). Esto la ha convertido, en lo que se refiere a una visión popular, en una disciplina de lujo accesible a unos pocos privilegiados económica o académicamente (agencias espaciales, grandes corporaciones, robots súper funcionalea, etc) llegando incluso a ser más cercana a la ciencia ficción que a la ingeniería corriente. No era vista como una disciplina científica cercana a la que poder dedicarse, mucho menos como una asignatura accesible a estudiantes jóvenes o universitarios. \\
Además, el  ``resultado'' de la robótica tampoco era visto como algo real, sino como ciencia ficción o productos de lujo económico. No se concebían los robots como algo adaptable a la vida corriente ni se pensaba en soluciones \textit{robóticas} a problemas reales. A esta visión de la robótica y de los robots ha contribuido en gran medida la literatura y el cine, creando en el imaginario popular robots humanoides indistinguibles de personas reales, completamente independientes y funcionales o inteligencias artificiales utilizadas para viajes espaciales que crecen y se desarrollan al margen de la humanidad.\\

Durante los últimos años, no obstante, se ha observado un cambio de paradigma, impulsado desde el mundo de la ingeniería y el pragmatismo comercial. Los robots han dejado de verse como ``humanos electrónicos'', convirtiéndose poco a poco en una definición más realista, como un proceso electrónico programado para cumplir una o varias funciones interactuando con el medio. Tenemos varios ejemplos de esta nueva definición de robótica:

\begin{description}
	\item [Automatismos.] También llamados \textit{\textbf{bots}}, son un robot sin una capa física que funciona de forma independiente recogiendo datos, procesándolos y respondiendo a ellos de forma \textit{inteligente}. Algunos ejemplos son los aplicativos de \textit{metadata} (muy extendidos) o los conocidos ``chat de ayuda'' en las aplicaciones web (asistentes de ayuda al cliente). Otro ejemplo de uso de estos automatismos son los procesos de seguridad de los vehículos modernos, que se mantienen recogiendo datos del medio  para ofrecer una capa añadida de seguridad frente a posibles errores humanos (velocidad, proximidad y velocidad de otros vehículos, distancias entre carriles, etc).\\
	Una de las características principales de estos \textit{procesos robotizados} es que arrancan sin requerir interacción. Aparte de que una vez funcionando respondan a peticiones o estímulos humanos, no requieren de que se les ``ordene'' encenderse. Siempre pueden apagarse, por motivos obvios de seguridad, pero volverán a arrancar con la misma configuración y, lo más importante, sin necesitar de ningún \textit{reset} o configuración inicial.
	
	\item [Internet Of Things.]\footnote{Internet de las cosas} Muy escuchado, aunque popularmente no se conecte con la robótica. La idea básica del IOT es añadirle conexión a Internet a objetos de uso diario, con los objetivos de:
	\begin{itemize}
		\item Añadir funcionalidades que requieren de conexión online. Por ejemplo, un reloj inteligente con el que poder atender mensajes o llamadas o pagar compras. El usuario interactúa con estas funcionalidades para obtener una respuesta, aunque la aplicación se mantiene funcionando siempre. Es decir, funciona esperando un estímulo al cual responder.
		\item Recoger datos para procesarlos y ofrecer información añadida en función de esos datos. El mismo reloj inteligente recoge datos de sueño, y te muestra la calidad y horas de éste. Con estas funciones no es necesario interactuar sino que están preparadas para recoger datos siempre que estén disponibles y mostrar al usuario los resultados de los procesos que tenga programados.
	\end{itemize}
	El IOT funciona con procesos robotizados, programados para recoger datos de forma automática o responder a estímulos del medio, que siguen las mismas normas antes mencionadas: se mantienen funcionando (siempre que no se les apague expresamente) sin necesidad de configuraciones añadidas; su programación se encarga de ello. Ciertamente, muchos objetos con IOT necesitan de datos iniciales para su puesta en marcha debido a que los datos recogidos son biométricos y necesitan de un contexto inicial para el procesamiento de datos. 
	
	\item [Domótica.] Posiblemente la utilidad robótica más extendida a día de hoy. Cada vez más unida al IOT, aunque no lo requiere expresamente. Viene del significado latino de 'casa' y engloba a todo sistema o automatismo diseñado para cumplir una función que facilite una tarea doméstica o le añada funcionalidad. Tenemos sistemas de seguridad, aspiradores, electrodomésticos que se comportan diferente dependiendo de las circunstancias (cantidad de agua de una lavadora, o temperatura de un frigorífico), control de luces con aplicaciones móviles (tanto con Internet como sin él), controles de accesos, etc. Todo esto son procesos robotizados (robots al fin y al cabo) que utilizan información del medio para reaccionar de una u otra manera. Esta información puede provenir de sensores (la lavadora o el frigorífico) o recibirla del usuario/a (apagar o encender las luces, la TV o la alarma).
	
	\item [Inteligencia Artificial.] Uno de los grandes trabajos respecto al cambio de visión de la robótica ha sido conseguir cambiar el ideario de Inteligencia Artificial y de separarlo del concepto de robot. Un robot puede contener un cierto grado de IA, pero no es necesario para el concepto de robótica. Una Inteligencia Artificial es un proceso robótico que, además de responder al medio, se adapta a él. Es decir, se realimenta con la información que va recogiendo del medio para ofrecer respuestas cada vez más precisas. Lejos del concepto que tradicionalmente se tenía de Inteligencia Artificial (como hemos comentado, en parte culpa de la ciencia ficción, por ejemplo el robot M.A.D.R.E. de la pelicula \textit{Alien}), podemos encontrar inteligencias artificiales habitualmente en nuestro día a día. Por ejemplo, los asistentes virtuales de diferentes fabricantes (Google, Amazon o Apple) aprenden del medio en mayor o menor medida y son capaces de buscar una respuesta que no tienen pre-programada, o los programas de procesamiento de \textit{big data} que se realimentan con los datos que recogen y de sus propios resultados para afinar esos mismos algoritmos de procesamiento.
	
	\item [Juguetes robots] Finalmente, llegamos a la aplicación más reconocible como ``robot'', también muy lucrativa. La robótica ha tenido un gran impulso en el ocio, creándose diferentes juguetes tanto para niños como para adultos. Se han convertido de ser un juguete de élite a tener variedad de opciones, con diversas funcionalidades, y accesibles a más variedad de costes. Esta accesibilidad de la robótica en el ocio ha contribuido en gran medida a esa transformación de la robótica y a incluirla en nuestras vidas. En cuanto a este Trabajo Fin de Grado, esta aplicación es la que más nos interesa.
\end{description}


Estos cambios por parte de la industria y avances por parte de la ingeniería han contribuido a acercar la robótica a las personas de a pie, sobre todo a las generaciones más jóvenes, facilitando el acceso a recursos con los que aprender y creando un interés por ello. Es muy notable el abaratamiento de los diversos componentes, así como de los productos finales, y la introducción de esta disciplina en los diferentes niveles educativos (sobre todo en estudios universitarios). El hecho que más gente tenga acceso a los recursos necesarios, y tenga desde joven los conceptos inculcados, contribuye a aumentar las aplicaciones prácticas en todos los ámbitos, y a resolver problemas en los que anteriormente, al no ser una disciplina extendida, quizá nadie había reparado. \\

Sin embargo, y a pesar de todo este trabajo realizado y todo lo conseguido, aún queda mucho por trabajar. El principal problema para que una persona se ponga a aprender cómo programar un \textit{comportamiento robótico} es que debe aprenderlo desde cero, sin haber tenido un aprendizaje progresivo como con el resto de enseñanzas. La Matemática, la Física o la Química, se enseña con niveles progresivos de dificultad, desde edades muy tempranas con las metodologías y conceptos ampliamente discutidos y formulados, para poder llegar a aplicaciones más avanzadas. Nadie espera que una persona sin conocimiento ninguno de estas materias, arranque  un curso de, por ejemplo, Cálculo Diferencial, Matemática Discreta o Teoría Gravitatoria. Es decir, se \textit{educa} al cerebro desde el principio, cuando más capacidad de aprendizaje se tiene, para adaptarse a nuevos conceptos y poder aprenderlos gradualmente. \\
\par Con la robótica y el pensamiento computacional no se ha seguido ese orden lógico. La mayoría de los estudiantes que aprenden Programación, Computación o Algoritmia lo hacen a edades más avanzadas, ya sea por su cuenta o con educación reglada. Si esta educación comenzara, como otras disciplinas, con los alumnos y alumnas siendo mucho más jóvenes y avanzara con ellos durante las siguientes etapas educativas, se conseguiría que llegaran a las más avanzadas mucho más preparados. Además despertaría en muchos más estudiantes la curiosidad por estas carreras, tanto profesional como educacionalmente.\\

\section{Propósito principal}
Con la idea de introducir en la robótica a niños y jóvenes como principal foco de atención, el propósito de este Trabajo Fin de Grado es ofrecer una propuesta para la enseñanza de robótica y programación a alumnos y alumnas sin conocimientos previos, tanto de Educación Primaria como Secundaria. Para ello se ha utilizado un robot educativo llamado mBot, del fabricante Makeblock \cite{makeblock}, que cumple los requisitos de simpleza y accesibilidad que requiere el trabajar con niños y niñas. Está especialmente preparado para la enseñanza sin dejar de ser un \textit{juguete}, ya que buscamos despertar el interés y convencer de la facilidad de una disciplina a menudo vista como especialmente complicada y aburrida.

Este proyecto estará compuesto de dos partes. La primera será la creación de una manera sencilla de programar el mBot en el lenguaje de programación Python, siendo éste mucho más sencillo que el lenguaje nativo del robot. Se explicará el proceso seguido para desarrollar esta solución, poniendo énfasis en detallar los pasos necesarios para replicarlo, utilizarlo en clases reales con alumnos e incluso ampliarlo. En la segunda parte encontraremos una propuesta completa de uso práctico de esta plataforma, con ejercicios detallados y soluciones de referencia, aumentando el nivel de dificultad progresivamente y con objetivos conceptuales claramente marcados.

Obtendremos así una forma para enseñar de forma accesible y fácil conceptos complejos como algoritmia, programación o metodologías de desarrollo de Software sin necesidad de explicaciones teóricas y poco adaptadas a edades alrededor de los 7-15 años.


\section{Antecedentes}\label{sec:antecedentes}

Dentro de la Universidad Rey Juan Carlos, el proyecto JdeRobot y particularmente el proyecto PyBoKids \cite{JdeRobot} han servido de antecedente ideológico a este Trabajo Fin de Grado. Ofrecían recursos para robótica educativa, teniendo en cuenta diferentes robot, diferentes plataformas y orientado a varios rangos de edades pre-universitarias.\\

En este Trabajo Fin de Grado hemos tomado del proyecto PyBoKids el propósito educativo, y el concepto de \textit{middleware entre el estudiante y el robot}. Durante la experiencia de un curso escolar de clases de robótica con alumnos de Educación Primaria se creó la necesidad de una solución educativa sencilla, lo más accesible posible en cuanto a herramientas y conceptos, y con objetivos marcados y preparados para que cualquiera pueda utilizarlos.



	%cada vez que quieras un capitulo nuevo, 
	%descomentarlo y crear el archivo .tex	
	\input{objetivos}
	\chapter{Infraestructura}\label{cap:infra}

\section{Scratch}\label{sec:scratch}
\section{Mbot}\label{sec:mbot}
\section{Python}\label{sec:python}
\section{Gazebo}\label{sec:gazebo}
	\chapter{Prácticas con Scratch}
\label{cap:scratch}
En este capítulo, se explicará cómo y por qué se diseñaron los distintos ejercicios con el lenguaje de programación educativo Scratch, haciendo hincapibooké en la finalidad de cada uno de ellos. \\
El capítulo se estructurá en varias secciones, una por cada área, u objetivo, de enseñanza: actuadores, sensores y comportamientos. Para cada sección, se explicarán los ejercicios que se han diseñado, con los propósitos que se buscaban con ese diseño (tanto para la robótica como para la programación en general) y la validación experimental que se ha hecho con los ejercicios en un entorno real.\\
El primer apartado hablará de los actuadores. La definición de un actuador es un componente que recoge datos de la placa (del programador, por tanto) y \textit{actúa} de acuerdo a esos datos. En ningún momento devuelve ningún dato a la placa. Por esta razón, la programación de este tipo de componente es más sencilla por ser más intuitiva: el alumno manda hacer algo al robot y el robot lo hace, sin necesidad del tratamiento de datos del entorno. Precisamente, al ser la programación del componente más sencilla en concepto, el diseño de ejercicios acepta más complejidad en conceptos de programación (conceptos como bucles, condiciones, esperas, etc). debido a esto se eligieron los actuadores como primer área para diseñar las prácticas.\\
La segunda sección de este capítulo se ocupará de los sensores. Un sensor recoge datos del entorno y los devuelve, en formato numérico, a la placa. Esto requiere de más control y estructuración de las ideas que los actuadores, pues requiere del programador llamar al sensor, guardar la información, tratarla de forma adecuada y, por último, programar la actuación del robot de acuerdo a ella. Por tanto, y contando con la base de los conocimientos y entrenamiento que habrían proporcionado los ejercicios del capítulo anterior, los sensores se escogieron como segundo área de conocimiento.\\
La tercera y última parte se centrará en la integración de los dos anteriores, es decir, en la integración de sensores y actuadores para la solución de problemas reales (todo lo reales que es posible, dadas las limitaciones del robot). Este tipo de ejercicios los llamamos comportamientos, y se han diseñado de tal forma que acepten escalonamiento en la dificultad. Esto responde a la necesidad de combatir la frustración de los alumnos más jóvenes, pues si se les presenta un problema sin dividir en problemas más pequeños, no sabrán cómo atajarlo. Obviamente, esta división del problema no se les dará a los alumnos directamente, sino que se les guiará para que la hagan ellos con el propósito de introducirlos en la programación \textit{bottom-up}.

\section{Actuadores}
Como se ha explicado anteriormente, el propósito de un actuador es ejecutar la orden que recibe de la placa. Es, por tanto, muy visual y conveniente para introducir en la programación de robótica, pues da pie a equivocarse y aprender de los errores de forma mucho más intuitiva que con otros componentes.\\
En la sección \ref{sec:mbot}, ya se detallaron los actuadores con los que cuenta el Mbot. En el diseño de estas prácticas no se han utilizado todos, pues se contó con la limitación de tener sólo el paquete "básico" del mbot (presumiblemente, sería el que tendrían los centros educativos). Estos componentes sería, pues, los motores, las luces led integradas en la placa, la placa led, el zumbador integrado y el pulsador integrado.\\
A continuación se explicarán los ejercicios desarrollados, intentando explicar el objetivo del ejercicio, el nivel de dificultad y comentando la experiencia de llevarlo a cabo.
\subsection{Carreras}
Obviamente, lo más visual del robot Mbot son los motores (las ruedas). También es lo más intuitivo y esperable de un robot en general: se mueve por sí mismo. Es uno de los componentes más sencillos de programar, ya que Scratch ofrece un desplegable con los posibles valores que aceptan los motores del Mbot. Además, tiene métodos directamente para avanzar, retroceder y girar, muy cómodos para el primer contacto con el robot y válidos para este ejercicio en concreto. \\
En vez de simplemente mover el robot, pensamos el ejercicio como una carrera entre los robot de todos los alumnos para despertar su interés y hacerlo más entretenido. Esto se comprobó necesario en la práctica; sin una utilidad práctica, o juego, los alumnos perdían interés rápidamente o, incluso, no lo tenían desde el principio. \\
\subsubsection{Ejercicio extra}
Otra cosa que se pudo comprobar cuando se pusieron en práctica los ejercicios es que es mejor hacer varias versiones de un mismo ejercicio, aunque la dificultad sea la misma, con objetivos distintos. Los alumnos no pierden la atención y aprenden que una misma solución, con pequeñas variaciones, vale para distintas situaciones. Además, así tenemos cubierta la situación de que los alumnos vayan a distintas velocidades y unos terminen el ejercicio base mientras otros no.\\
Una versión del ejercicio de las carreras es hacerla con obstáculos fijos. Los alumnos tendrán que medir en tiempo la distancia entre obstáculos y jugar, con las velocidades y los giros. Quién mejor combinación entre ello encuentre, mejor tiempo hará en carrera.

\subsection{Cumpleaños feliz}
Otro actuador sencillo de usar es el zumbador integrado en la placa. Es algo más complicado, pues las notas musicales no son las más conocidas y, sobre todo en los alumnos más jóvenes, no están acostumbrados a una escala musical. Sin embargo, no puede comprobar que no tardan mucho en entender el sistema de escalas; se les puede ayudar dándoles una tabla de equivalencias entre las notas americanas y europeas. 
\begin{figure}[H]
	\centering
	\includegraphics[width=0.7\textwidth]{img/lorem.jpg}
	\caption{Equivalencia entre notas}	
\end{figure}
Una vez entendida la diferencia de notas y comprobado cómo funciona el zumbadores, les proponemos que el robot entone la canción de cumpleaños feliz -por ser una canción que todo el mundo conoce-. No sólo sirve para el actuador, sino que introducimos el concepto de repetición (bucles). La idea principal es dejarles al principio que hagan el ejercicio todo seguido y después les ayudamos a entender que están repitiendo código: hay estrofas que se repiten. Ven la necesidad, pues, de meter esas estrofas dentro de un bucle de repetición de tantas veces como sea necesario.\\
\subsubsection{Ejercicio extra}
Como ocurrió antes con las carreras, una vez han terminado la primera canción, es bueno invitarles a que intenten que el robot "cante" otras melodías que conozcan y les apetezca. 

\subsection{S.O.S.}
El siguiente actuador del pack del Mbot son las luces LED de la placa, que tienen un rango de colores RGB que permite que cada alumno pueda personalizar los colores como más les guste. El propio lenguaje Scratch permite el rango de cada color entre 0 y 255 (aunque en la práctica, no sean notables los cambios pequeños en esa escala).\\
La finalidad del ejercicio es que, con luces, el robot pida ayuda en lenguaje morse: tres pulsos cortos, tres pulsos largos, y tres pulsos cortos. \\
Por experiencia, son más proactivos cuando ven una utilidad mínimamente real a lo que se les pide, y en muchas películas han visto hablar del morse y la necesidad de pedir ayuda en un lenguaje universal. \\
Como concepto de programación, en este ejercicio trabajamos las esperas. Los puntos y rayas del morse los traducimos a esperas de menos y más tiempo, respectivamente.
bucle de repetición de tantas veces como sea necesario.\\
\subsubsection{Ejercicio extra}
Como siempre, una vez que han conseguido que el mensaje de socorro sea entendible, cambiamos el ejercicio a escribir en morse cualquier mensaje que se les ocurra (con la tabla de equivalencias del abecedario) y a mandar el mismo mensaje de socorro con luz y sonido a la vez.

\subsection{Camión}
Como último ejercicio de actuadores, intentaremos juntarlos todos, a la vez que todavía no metemos ningún sensor. Proponemos, pues, emular el comportamiento de un camión, algo a lo que están acostumbrados y que no se les habría ocurrido que era automatizable. Un camión, haga lo que haga, si da marcha atrás, pitará (con los zumbadores) y encenderá las luces naranjas. \\
Vamos metiendo, poco a poco, acciones simultáneas, para que tengan que preocuparse de más de un componente (de más de un bloque de Scratch). El objetivo es el aprendizaje de que, en robótica, un comportamiento aparentemente trivial es la suma de muchos comportamientos simultáneos y dependiente uno de otro.
	\chapter{Prácticas con Python y Mbot-real}
\label{cap:real}
	\chapter{Prácticas con Python y Mbot-simulado}
\label{cap:simulado}
	\input{conclusiones}
	
	\bibliographystyle{acm} %para que se muestre la bibliografía
	\bibliography{bibliografia_tfg.bib}
	%En la bibliografía no sale nada que no se haya citado antes (se supone que si no lo citas es porque no lo has usado)
	
\end{document}